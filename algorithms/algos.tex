Heuristics \ra efficient algos that get a good although not necessarily perfect solution
\chapter{Algorithms}
\section{Asymptotics}
How the algorithm grows as N \ra $\infty$.

\rsrc{Algorithms by David Sedgewick} Page 67

\rsrc{David Mount Notes}\href{http://www.cs.umd.edu/~mount/251/Lects/251lects.pdf}{Link}

\rsrc{Big-O Notation}\href{https://en.wikipedia.org/wiki/Big_O_notation}{Wiki Link}

\rsrc{Time Complexity}\href{https://en.wikipedia.org/wiki/Time_complexity#Constant_time}{Wiki Link}

\subsection{Notation}
	\textbf{$\Theta$}: The asymptotic class of an algo. 
	\begin{equation}
		\Theta(g(n)) \equiv \bigg\{f(n), 0<=c_1g(n)<=f(n)<=c_2g(n) | c_1, c_2, n_0 \in |\Re| \textrm{ and } n_0 <= n \bigg\}
	\end{equation}
	For a algo to be $\Theta$(g(n)) it needs to be both O(g(n)) and $\Omega$(g(n)). A $\Theta$(g(n)) grows at exactly g(n).

	\hspace{11pt}
	
	\textbf{O}: Upper asymptotic bound for an algo. An algo that has O(g(n)) grows at or slower than that rate. 
		\begin{equation}
			O(g(n)) \equiv \bigg\{f(n) | 0<= f(n) <=cg(n) | c, n_0 \in |\Re| \textrm{ and } n_0 <= n \bigg\}
		\end{equation}

	\hspace{11pt}

	\textbf{$\Omega$}: The lower bound on the growth. An algo w/ $\Omega$(g(n)) grows at or faster than g(n). 
		\begin{equation}
			\Omega(g(n)) \equiv \bigg\{f(n) | 0 <= cg(n) <= f(n) | c_1, c_2, n_0 \in |\Re| \textrm{ and } n_0 <= n \bigg\}
		\end{equation}

		\subsection{Performace}
		\subsubsection{Worst Case}
		\subsubsection{Best Case}
		\subsection{Average Case}

\subsection{Asymptotic Analysis}
\subsubsection{(Strong) Induction}
\subsubsection{Iteration}
\subsubsection{Recurrance}

\subsection{Master's Theorem}


\begin{table}
	\caption{Asymptotic growth types}
	\label{table:asymtoticGrowth}
\begin{tabular}{lll}
\hline\hline
Notation& Name& Example\\\hline
O(1)&Constant&Seeing if a binary number is even or odd\\
O(log log n) &Double Logarthmic& \\
O(log n)& Logarithmic& Finding and item in a sorted array with binary search\\
O( (log n)$^c$ ) w/ $c>1$& Polylogarithmic& \\
O(n$^c$) w/ $0<c<1$&Fractional power &Searching in a kd-tree \\
O(n)& Linear& Find an item in an unsorted list\\
O(n log*n)& n log-star n& Union-find\\
O(n log n)& quasilinear/linearithmic& FFT\\
O(n$^2$)& Quadratic&Common limit on sorting\\
O(n$^c$)& Polynomial& LU decomposition\\
\hline\hline
\end{tabular}
\end{table}

\section{Algorithm Types}
\subsection{Divide \& Conquer}
\subsection{Selection}

\section{Searching}
\subsection{Binary Search}

\rsrc{Binary Search}\href{https://en.wikipedia.org/wiki/Binary_search_algorithm}{Wiki Link}

\subsection{Linear Search}

\rsrc{Linear Search}\href{https://en.wikipedia.org/wiki/Linear_search}{Wiki Link}

\section{Selection}
\subsection{Sieve Technique}

\section{Sorting}
\subsection{Merge Sort}
\subsection{Heap Sort}
\subsection{Quick Sort}
\subsection{Bubble Sort}
\subsection{Insertion Sort}
\subsection{Selection Sort}
\subsection{Count Sort}
\subsection{Radix Sort}
