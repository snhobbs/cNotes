\chapter{Object Oriented Programming}

\section{Design Patterns}

\begin{table}
\caption{Creation Patterns}
\begin{tabular}{rp{.8\textwidth}
Abstract Factory&\\
Builder&\\
Factory Method&\\
Prototype&\\
Singleton&\\
\end{tabular}
\end{table}

\begin{table}
\caption{Structural Patterns}
\begin{tabular}{rp{.8\textwidth}
Adaptar&\\
Bridge&\\
Composite&\\
Decorator&\\
Facade&\\
Flyweight&\\
Proxy&\\
\end{tabular}
\end{table}

\begin{table}
\caption{Behavioral Patterns}
\begin{tabular}{rp{.8\textwidth}
Chain of Responsibility&\\
Command&\\
Interpreter&\\
Iterator&\\
Mediator&\\
Memento&\\
Observer&\\
State&\\
Strategy&\\
Template Method&\\
Visitor&\\
\end{tabular}
\end{table}

	\begin{questions}{OOP Terms}
		\begin{questionAnswer}
			\qItem{Singleton}{Design Pattern that prevents multiple instantiation}
			\qItem{Borg}{Multiple intances share state}
			\qItem{First class objects}{Objects can be used in the same way other data types can, passed to functions assigned }
			\qItem{Super/Base/Parent}{The class/es that a derived/child/sub/child class inherits from. All 3 are synonymous}
			\qItem{derived/child/sub/child}{ The classes that inherit from a Super/Base/Parent class}
			\qItem{Overriding}{A derived class redefines a method that exists in the base class, the derived class's method is then called instead}
			\qItem{HAS-A}{}
			\qItem{IS-A}{}
		\end{questionAnswer}
	\end{questions}

	\begin{questions}{Python OOP}
		Python is polymorphic, everything-is-an-object, multi-inheriting,
	\end{questions}

	\begin{questions}{Terms}
		\begin{questionAnswer}
			\qItem{Memoization}{Saving the result of a function call so as to skip recalculation on repeated calls}
			\qItem{Polymorphic}{Single interface to different types.}
		\end{questionAnswer}
	\end{questions}
